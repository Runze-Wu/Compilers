\documentclass[11pt]{article}
\usepackage{bm}
\usepackage{ctex}
\usepackage{array}
\usepackage{color}
\usepackage{float}
\usepackage{xcolor}
\usepackage{amssymb}
\usepackage{amsmath}
\usepackage{diagbox}
\usepackage{graphicx}
\usepackage{listings}
\usepackage{tabularx}
\usepackage{hyperref}
\usepackage{fancyhdr}
\usepackage{enumitem}
\usepackage{enumerate}
\definecolor{keywordcolor}{rgb}{0.8,0.1,0.5}

\pagestyle{fancy}
\hypersetup{hypertex = true,
			colorlinks = false,
			linkcolor = blue ,
			anchorcolor = blue,
			citecolor = blue}
\fancyhf{}


\definecolor{mygreen}{rgb}{0,0.6,0}  
\definecolor{mygray}{rgb}{0.5,0.5,0.5}  
\definecolor{mymauve}{rgb}{0.58,0,0.82}  

\lstset{ %  
	backgroundcolor=\color{white},   % choose the background color; you must add \usepackage{color} or \usepackage{xcolor}  
	basicstyle=\footnotesize,        % the size of the fonts that are used for the code  
	breakatwhitespace=false,         % sets if automatic breaks should only happen at whitespace  
	breaklines=true,                 % sets automatic line breaking  
	captionpos=bl,                    % sets the caption-position to bottom  
	commentstyle=\color{mygreen},    % comment style  
	deletekeywords={...},            % if you want to delete keywords from the given language  
	escapeinside={\%*}{*)},          % if you want to add LaTeX within your code  
	extendedchars=true,              % lets you use non-ASCII characters; for 8-bits encodings only, does not work with UTF-8  
	frame=single,                    % adds a frame around the code  
	keepspaces=true,                 % keeps spaces in text, useful for keeping indentation of code (possibly needs columns=flexible)  
	keywordstyle=\color{blue},       % keyword style  
	%language=Python,                 % the language of the code  
	morekeywords={*,...},            % if you want to add more keywords to the set  
	numbers=right,                    % where to put the line-numbers; possible values are (none, left, right)  
	numbersep=5pt,                   % how far the line-numbers are from the code  
	numberstyle=\tiny\color{mygray}, % the style that is used for the line-numbers  
	rulecolor=\color{black},         % if not set, the frame-color may be changed on line-breaks within not-black text (e.g. comments (green here))  
	showspaces=false,                % show spaces everywhere adding particular underscores; it overrides 'showstringspaces'  
	showstringspaces=false,          % underline spaces within strings only  
	showtabs=false,                  % show tabs within strings adding particular underscores  
	stepnumber=1,                    % the step between two line-numbers. If it's 1, each line will be numbered  
	stringstyle=\color{orange},     % string literal style  
	tabsize=2,                       % sets default tabsize to 2 spaces  
	%title=myPython.py                   % show the filename of files included with \lstinputlisting; also try caption instead of title  
}

\chead{Compiler}
\fancyhead[r]{\bfseries\thepage}
\fancyhead[l]{\bfseries\rightmark}
\setlength{\listparindent}{0em} %段落缩进量
\begin{titlepage}
	\title{06-02编译原理作业14}
	\author{
			\textbf{作者:} {吴润泽}
			\and {\textbf{学号:} 181860109}
		}
\end{titlepage}
\begin{document}
\maketitle
\section*{9.2.1}
\subsection*{1)}
\begin{table}[H]
\centering
\begin{tabular}{|c|c|c|}
	\hline
	\textbf{基本块} & \textbf{gen} & \textbf{kill} \\ \hline
	$B_1$ & 1,2 & 8,10,11 \\ \hline
	$B_2$ & 3,4 & 5,6 \\ \hline
	$B_3$ & 5 & 4,6 \\ \hline
	$B_4$ & 6,7 & 4,5,9 \\ \hline
	$B_5$ & 8,9 & 2,7,11 \\ \hline
	$B_6$ & 10,11 & 1,2,8 \\ \hline
\end{tabular}
\end{table}
\subsection*{2)}
\begin{table}[H]
\centering
\begin{tabular}{|c|c|c|}
	\hline
	\textbf{基本块} & \textbf{IN} & \textbf{OUT} \\ \hline
	$B_1$ & $\emptyset$ & 1,2 \\ \hline
	$B_2$ & 1,2,3,5,8,9 & 1,2,3,4,8,9 \\ \hline
	$B_3$ & 1,2,3,4,6,7,8,9 & 1,2,3,5,7,8,9 \\ \hline
	$B_4$ & 1,2,3,5,7,8,9 & 1,2,3,6,7,8 \\ \hline
	$B_5$ & 1,2,3,5,7,8,9 & 1,3,5,8,9 \\ \hline
	$B_6$ & 1,3,5,8,9 & 3,5,9,10,11 \\ \hline
\end{tabular}
\end{table}
\section*{9.2.3}
\subsection*{1)}
\begin{table}[H]
\centering
\begin{tabular}{|c|c|c|}
	\hline
	\textbf{基本块} & \textbf{def} & \textbf{use} \\ \hline
	$B_1$ & a,b & $\emptyset$ \\ \hline
	$B_2$ & c,d & a,b \\ \hline
	$B_3$ & $\emptyset$ & b,d \\ \hline
	$B_4$ & d & a,b,e \\ \hline
	$B_5$ & e & a,b,c \\ \hline
	$B_6$ & a & b,d \\ \hline
\end{tabular}
\end{table}
\subsection*{2)}
\begin{table}[H]
\centering
\begin{tabular}{|c|c|c|}
	\hline
	\textbf{基本块} & \textbf{gen} & \textbf{kill} \\ \hline
	$B_1$ & e & a,b,e \\ \hline
	$B_2$ & a,b,e & a,b,c,d,e \\ \hline
	$B_3$ & a,b,c,d,e & a,b,c,d,e \\ \hline
	$B_4$ & a,b,c,e & a,b,c,d,e \\ \hline
	$B_5$ & a,b,c,d & a,b,d \\ \hline
	$B_6$ & b,d & $\emptyset$ \\ \hline
\end{tabular}
\end{table}
\end{document}