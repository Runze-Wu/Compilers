\documentclass[11pt]{article}
\usepackage{bm}
\usepackage{ctex}
\usepackage{array}
\usepackage{color}
\usepackage{float}
\usepackage{xcolor}
\usepackage{amssymb}
\usepackage{amsmath}
\usepackage{diagbox}
\usepackage{graphicx}
\usepackage{listings}
\usepackage{tabularx}
\usepackage{hyperref}
\usepackage{fancyhdr}
\usepackage{enumitem}
\usepackage{enumerate}
\definecolor{keywordcolor}{rgb}{0.8,0.1,0.5}

\pagestyle{fancy}
\hypersetup{hypertex = true,
			colorlinks = false,
			linkcolor = blue ,
			anchorcolor = blue,
			citecolor = blue}
\fancyhf{}


\definecolor{mygreen}{rgb}{0,0.6,0}  
\definecolor{mygray}{rgb}{0.5,0.5,0.5}  
\definecolor{mymauve}{rgb}{0.58,0,0.82}  

\lstset{ %  
	backgroundcolor=\color{white},   % choose the background color; you must add \usepackage{color} or \usepackage{xcolor}  
	basicstyle=\footnotesize,        % the size of the fonts that are used for the code  
	breakatwhitespace=false,         % sets if automatic breaks should only happen at whitespace  
	breaklines=true,                 % sets automatic line breaking  
	captionpos=bl,                    % sets the caption-position to bottom  
	commentstyle=\color{mygreen},    % comment style  
	deletekeywords={...},            % if you want to delete keywords from the given language  
	escapeinside={\%*}{*)},          % if you want to add LaTeX within your code  
	extendedchars=true,              % lets you use non-ASCII characters; for 8-bits encodings only, does not work with UTF-8  
	frame=single,                    % adds a frame around the code  
	keepspaces=true,                 % keeps spaces in text, useful for keeping indentation of code (possibly needs columns=flexible)  
	keywordstyle=\color{blue},       % keyword style  
	%language=Python,                 % the language of the code  
	morekeywords={*,...},            % if you want to add more keywords to the set  
	numbers=right,                    % where to put the line-numbers; possible values are (none, left, right)  
	numbersep=5pt,                   % how far the line-numbers are from the code  
	numberstyle=\tiny\color{mygray}, % the style that is used for the line-numbers  
	rulecolor=\color{black},         % if not set, the frame-color may be changed on line-breaks within not-black text (e.g. comments (green here))  
	showspaces=false,                % show spaces everywhere adding particular underscores; it overrides 'showstringspaces'  
	showstringspaces=false,          % underline spaces within strings only  
	showtabs=false,                  % show tabs within strings adding particular underscores  
	stepnumber=1,                    % the step between two line-numbers. If it's 1, each line will be numbered  
	stringstyle=\color{orange},     % string literal style  
	tabsize=2,                       % sets default tabsize to 2 spaces  
	%title=myPython.py                   % show the filename of files included with \lstinputlisting; also try caption instead of title  
}

\chead{Compiler}
\fancyhead[r]{\bfseries\thepage}
\fancyhead[l]{\bfseries\rightmark}
\setlength{\listparindent}{0em} %段落缩进量
\begin{titlepage}
	\title{05-26编译原理作业13}
	\author{
			\textbf{作者:} {吴润泽}
			\and {\textbf{学号:} 181860109}
		}
\end{titlepage}
\begin{document}
\maketitle
\section*{8.6.5}
\begin{lstlisting}
1)	t1	=	a
	LD	R1,	a
2)	t2	=	b	*	c
	LD	R2,	b
	LD	R3,	c
	MUL	R2,	R2,	R3
3)	t3	=	t1	+	t2
	ADD	R1,	R1,	R2
4)	x	=	t3
	ST	x,	R1
\end{lstlisting}
\begin{table}[H]
	\centering
	\resizebox{\textwidth}{!}{%
		\begin{tabular}{|c|c|c|c|c|c|c|c|c|c|c|}
			\hline
			\diagbox{\textbf{执行代码数目}}{\textbf{描述符状态}} 
			&
			\textbf{R1} &
			\textbf{R2} &
			\textbf{R3} &
			\textbf{a} &
			\textbf{b} &
			\textbf{c} &
			\textbf{x} &
			\textbf{t1} &
			\textbf{t2} &
			\textbf{t3} \\ \hline
			0 &      &    &   & a    & b & c    & x  &    &    &    \\ \hline
			1 & a,t1 &    &   & a,R1 & b & c    & x  & R1 &    &    \\ \hline
			2 &      & t2 & c & a    & b & c,R3 & x  & R1 & R2 &    \\ \hline
			3 & t3   & t2 & c & a    & b & c,R3 & x  &    & R2 & R1 \\ \hline
			4 & t3,x & t2 & c & a    & b & c,R3 & R1 &    & R2 & R1 \\ \hline
		\end{tabular}
	}
\end{table}
\section*{9.1.1}
\begin{itemize}
	\item[1)]
	$B_3,B_4$是循环,$B_2,B_3,B_4,B_5$是循环.
	\item[2)]
	a可以使用常量进行替换;b不可以.
	\begin{itemize}
	\item[(3)]	c=1+b
	\item[(4)]	d=c-1
	\item[(6)]	d=1+b
	\item[(8)]	b=1+b
	\item[(9)]	e=c-1
	\end{itemize}
	\item[3)]
	对于$B_3,B_4$循环, a+b是公共子表达式;{\color{red}(x)}\\
	对于$B_2,B_3,B_4,B_5$循环, a+b, c-a是公共子表达式.
	\item[5)]对于$B_3,B_4$循环, a+b是循环不变计算;\\
	对于$B_2,B_3,B_4,B_5$循环,没有循环不变计算.
\end{itemize}
\section*{9.1.4}
\begin{lstlisting}
	dp	=	0
	i	=	0
	t1	=	0
L:t2	=	A[t1]
	t4	=	B[t1]
	t5	=	t2	*	t4
	dp	=	dp	+	t5
	i	=	i	+	1
	t1	=	t1	+	8
	if	i	<	n	goto	L
\end{lstlisting}
{\large{\color{red} 归纳变量强度消减}}
\end{document}